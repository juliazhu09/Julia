\chapter{Introduction} \label{ch:introduction}
\section{Problem Statement}
Employee turnover refers to an employee's leaving his/her job as a result of voluntary quitting, retirement, disability, or death. Employee turnover has drawn management researchers' and practitioners' attention for decades because turnover cost affects an organization's operational capabilities and budget. Employee turnover is both costly and disruptive to the organizational function \citep{kacmar2006, mueller1989,staw1980}; and both private firms and governments spend billions of dollars every year managing this issue \citeyearpar{leonard2001}. %check reference
Turnover costs involve recruiting, selecting, and training \citep{mobley1982, staw1980}. According to the U.S. Department of Labor, turnover costs a company one third of a new hire's annual salary to replace an employee, which is about \$500 to \$1500 per person for the fast-food industry and \$3000 to \$5000 per person for the trucking industry \citep{white1995}. 

Furthermore, turnover disrupts social and communication structures and causes productivity loss \citep{mobley1982}. Turnover also demoralizes the remaining employees and leads to additional turnover \citep{staw1980}. \citet{sagie2002} found that a high-tech firm lost 2.8 million US dollars or 16.5\% of before-tax annual income because of employee turnover. These researchers also found that turnover reduced profits, increased the organization's total risk, and triggered more turnover among the organization's other employees. Therefore, understanding and forecasting turnover at firm and departmental levels is essential for reducing it \citep{kacmar2006}, as well as for effectively planning, budgeting, and recruiting in the human resource field. 

Many studies have shown that employee turnover significantly affects an organization's performance. \citet{staw1980} summarized previous studies indicating that turnover reduces organizations' team or work performance and financial performance and significantly changes an organization's direction when a top executive leaves. \citet{glebbeek2004} found that excessive employee turnover harms firm performance (profits). According to \citet{hancock2013}'s study, turnover has a strong negative relationship with organizational performance in the manufacturing and transportation industries. According to \citet{kacmar2006}'s study, employee turnover increases customer waiting time and reduces profits. Turnover also reduces restaurants' profitability and customer satisfaction because of declining productivity \citep{detert2007}. On the other hand, the lower turnover rate increases sales \citep{batt2002}. Overall, the inability to predict employee turnover and to replace that individual reduces organizational performance and profits and disrupts the organizational structure.

Predicting employee turnover helps reduce the hiring lead time, thus eliminating some turnover costs. The lead time of employee turnover and replacement involves five stages: providing a leaving notice, advertising the job opening, interviewing, doing a background check, and completing on-board training. An employee usually provides a leaving notice at least two weeks before the actual leaving date. Then the employee's manager must prepare for advertisement and hiring committees, requiring a week to complete.  The hiring committees need two to eight weeks, sometimes even longer, to release the hiring advertisement and select interview candidates. During the interview period, hiring committees need one to two weeks to make a final decision. One week to six months is needed to complete the finalist's background security check. Finally, the employee's on-board training takes at least one week. Thus, an organization usually spends at least two months replacing a new employee. For some governmental organizations, the hiring lead time is much longer since their security check takes three to six months. The forecasting system provides the predicted turnover number a year in advance. Based on a combination of this number, the production plan, and the budget, the HR department determines a final demand number. Ignoring employees' notice periods, HR can either advertise job openings and start hiring immediately based on the final demand number (thus reducing the training period) or wait until an employee provides the leaving notice.  Therefore, forecasting turnover at firm and departmental levels reduces hiring lead time \citep{kacmar2006}. 
\section{Purpose of study}
Some organizations have a long hiring lead time because of the time required for background security checks ($>$ 6 months). This study developed a methodology to forecast employee turnover at organizational and departmental levels to shorten this time. It also investigated turnover seasonal patterns and such factors as employee demographics, job categories, and organization structures; identified influential financial indexes; and measured the magnitude of retention and the Early Retirement Incentive Program (ERIP) \citep{ERIP} that HR released; and forecasted employees' turnover based on retirement and voluntary quitting. The study also simulated turnover datasets to measure two data biases and examines the forecasting models' forecasting capability. Finally, the study examined job title, gender, ethnicity, age, and years of service to assist in employee-retention strategies to reduce R\&D departments' voluntary turnover rate. 
\section{Approach}
Forecasting employee turnover is a crucial part of a lean management system \citep{allway2002}. Also called the Toyota production system, lean management involves operating the most efficient and effective organization possible with the least cost and {\it zero} wastes \citep{jackson1996}. The waste is non-value-added activities. High employee turnover is one kind of waste \citep{kilpatrick2003}. According to \citet{allway2002} ), employee turnover is a significant factor in lean management. 

An employee-turnover forecasting system reduces wastes that high turnover causes. By identifying factors influencing employee turnover, the forecasting system assists HR in determining retention strategies to reduce turnover rates. According to \citet{yeung1997} and \citet{kochan1992}, "high-commitment" human resource polices contribute to a lean management system's success.

Furthermore, an employee-turnover forecasting system reduces the setup time of the organization's lean-management system. According to \citet{lin1999}, employee turnover takes a lean organization much more time and far more resources. Productivity is usually reduced because an employee leaves and the replacement is unfamiliar with the work. The new employee needs a training period to be as productive as the previous employee. To prevent productivity reduction, HR could hire a new replacement based on the demand number computed from employee-turnover forecasting. Then a new employee could receive job training before the old employee leaves the position. As a result, productivity level could remain the same before and after employee turnover.  

Finally, an employee-turnover forecasting system periodically provides current and future inventory levels. The inventory management system uses the demand number the system provides, the hiring lead time, and the hiring costs to determine the optimal hiring number through an appropriate economic order model. As a result, the skill-set inventory is systematically controlled.

%On other hand, the application of lean management system can reduce the employee turnover. \citet{jimmerson2005} summarized that companies applied lean manufacturing can improved productivity, increased customer satisfaction and greatly reduced employee turnover in the literature \citep{liker1997, womack2000}. \citet{laureani2010} applied lean six sigma to the human resource function of a service company and found that their application reduced employee voluntary turnover and increased their satisfaction, thus increasing the return on investment of human capital. 

\section{Methodology}
%%%%% need to decide this part is at approach or methodogy.
Statistical methods are used to develop employee-turnover forecasting models. These methods are time series, survival analysis, logistic regression, and data mining. Time series methods capture employee turnover's seasonal and cyclical patterns and forecast an aggregated turnover number, in terms of headcount, by using a historical turnover number. Survival analysis identifies significant internal and external turnover factors and builds a Cox PH model to forecast turnover at the individual, departmental, and entity-wide levels. Logistic regression and decision tree methods determine whether significant factors identified in the literature are also significant in the organization studied. A set of decision rules are created based on employees' tenure. 
   
Implementing an employee-turnover forecast model is also a key part of workforce planning in a lean management system. The model is inserted into a software program using a user-friendly interface. The human resource department installs this program, imports the employee's information into the program periodically, computes forecasting information, and exports  the results. Based on the results, HR either modifies the employee retention and promotion strategies for the targeted employees with high turnover probabilities to reduce the employee turnover rate or prepares to hire new employees to prevent reduced productivity and lean-management system malfunction.


\section{Outline}
This dissertation includes five chapters. Chapter 1 introduces employee-turnover forecasting. Chapter 2 introduces time series methods and identifies an optimal time series model to forecast the monthly employee-turnover number. Chapter 3 discusses how to build a Cox proportional model to identify the significant demographics, organizational factors, and financial indexes as well as to forecast employee retirement and voluntary quitting on an individual level. Chapter 4 presents the hypotheses regarding employee turnover significant factors for a R\&D organization and splitting rules used to forecast the average tenure. Chapter 5 summarizes the findings, the implementation of forecasting model, and future work.

