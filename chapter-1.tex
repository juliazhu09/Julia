\chapter{Introduction} \label{ch:introduction}
\section{Problem Statement}
Employee turnover has drawn management researchers and practitioners attentions for decades, because employee turnover cost impacts both the operational capabilities and the budget of an organization. Employee turnover is both costly and disruptive to the functioning of most organizations \citep{kacmar2006, mueller1989,staw1980}, and both private firms and governments spend billions of dollars every year managing the issue according to Leonard \citeyearpar{leonard2001}. %check reference
Employee turnover is an employee leaving his current working organization due to variety reasons, such as voluntary quitting, retirement, disability, or death.
The turnover cost involves recruiting, selecting, training and developing \citep{mobley1982, staw1980}. According to the estimation from U.S. Department of Labor, turnover costs a company one third of a new hire's annual salary to % check this sentence
replace an employee, which is about \$500 to \$1500 per person for fast-food industry and \$3000 to \$5000 per person for trucking industry \citep{white1995}. 

Furthermore, turnover also disrupts the social and communication structures, and causes the productivity loss due to the replacement \citep{mobley1982}. Beyond the replacement cost and operational disruption, turnover demoralizes the attitudes of remaining employees and leads to additional turnover \citep{staw1980}. \citet{sagie2002} found that 2.8 million US dollars or 16.5\% of the before-tax annual income was lost for a high tech firm due to the withdrawal behaviors and they concluded that turnover reduced the profit, increased the total risk of organizations, and triggered more turnover for the other employees in the organization. Therefore, understanding and forecasting turnover at firm and departmental levels is essential for reducing it \citep{kacmar2006} and for effective planning, budgeting, and recruiting in the human resource field. 

Many studies have proved that employee turnover has significant effects on organizational performance. \citet{staw1980} summarized the previous studies that turnover reduced the team or work performance and financial performance of organizations, and significantly changed the direction of an organization when top executives leaves. \citet{glebbeek2004} found that overly high employee turnover was harmful for firm performance (profits). According to \citet{hancock2013}'s study, turnover has strong negative relationship with organizational performance in manufacturing and transportation industries. 
According to \citet{kacmar2006}'s study employee turnover caused customer waiting time increasing and reduced the store profits. Turnover also reduces the restaurant profitability and customer satisfaction due to counter productivity declining \citep{detert2007}. On the other hand, the lower turnover rate increase the sales \citep{batt2002}. Overall, unable to predict employee turnover and to replace that individual reduce organizational performance and profits and disrupt the organizational structure. 

Employee turnover prediction help reduce the hiring lead time, and as a results of eliminating the some turnover costs. The lead time of employee turnover and replacement includes six stages, such as, notice, approval, advertisement, interview, background check, and on-board training. An employee usually provides a leaving notice at lease two weeks before the actual leaving date; the employee's manager has approval it and prepare for advertisement and hiring committees (one week); the hiring committees needs two to eight weeks, sometimes even longer, to release the hiring advertisement and select interview candidates; the interview period takes hiring committees one to two weeks to make a final decision; the finalist needs one week to six months background security check; after background check, the employee's on-board and training takes at lease one week to six months. Clearly, a organization usually spends at least two months to replace a new employee. For some governmental organization, the hiring lead time is much longer since their security check takes from three months to six months. Employee turnover forecasting shorten the hiring lead time. The forecasting system provides the predicted turnover number in a year ahead. The HR department determines a final demand number through considering this number and its budget. Regardless employees' notice and approval periods, it can release its advertisement and start to hire immediately based on the final demand or right after receiving a leaving notice. Therefore, understanding and forecasting turnover at firm and departmental levels is essential for reducing it \citep{kacmar2006} and for effective functional lean management system. 
\section{Purpose of study}
The studied governmental organizations have a very long lead time to hiring a new employee, due to the long background security check usually more than 6 months. This study developed a methodology to forecast employee turnover in organizational level to shorten employee hiring lead time. It also investigates turnover seasonal patterns and factors, such as employee demographic information, job categories, and orientations structures, identifies influential financial indexes, measures the magnitude of retention and Early Retirement Incentive program \citep{ERIP} released by (HR) department, forecasts employees's turnover by two causes, retirement and voluntary quitting. The study also simulates the turnover datasets to measure two data biases and examine the forecasting capability of the forecasting models.   
\section{Approach}
Employee turnover forecasting effectively assists the operation of lean management system. Actually, it is crucial part of lean management system \citep{allway2002}. Lean management system, also called toyota production system, is about operating the most efficient and effective organization possible, with the least cost and {\it zero} wastes \citep{jackson1996}. \citet{allway2002} claimed that the employee turnover is significant factors for lean tools. The waste is non-valued-added activities. The high employee turnover are one kind of wastes \citep{kilpatrick2003}. 

Employee turnover forecasting system reduce the wastes causing by the high turnover. It identifies the factors of employee turnovers. The influential factors assists HR to determine specific retention strategies focusing on target employee groups to reduce the turnover rates. According to \citet{yeung1997} and \citet{kochan1992}, "high-commitment" human recourse polices successes the lean management system.   

Further more, Employee turnover forecasting system reduce the setup time of the organization lean management system. \citet{lin1999} claimed that employee turnover leads to lean organization take much more time and far more resources to select and train a new employee who fits the position. Usually, the productivity is reduced when a employee leaving and the new replacement is not familiar his works yet. The new employee needs a period to reach the same productivity as the previous employee. Employee turnover forecasting reduces the hiring lead time by eliminating the hiring stages. Therefore, the new employee get trained and be familiar his work before the old employee leaving the position, as a results the organization productivity remains at the same level.  

Finally, Employee turnover forecasting system assist the organization talent and skill set inventory control. It periodically provides the current and future inventory level. To keep the inventory on a certain level. the hiring manager has to use the demand number provided by the system, the hiring lead time, and hiring costs to determine the optimal hiring number through a appropriate economic order model. As a results, the skill set inventory is systematically controlled.

%On other hand, the application of lean management system can reduce the employee turnover. \citet{jimmerson2005} summarized that companies applied lean manufacturing can improved productivity, increased customer satisfaction and greatly reduced employee turnover in the literature \citep{liker1997, womack2000}. \citet{laureani2010} applied lean six sigma to the human resource function of a service company and found that their application reduced employee voluntary turnover and increased their satisfaction, thus increasing the return on investment of human capital. 

\section{Methodology}
%%%%% need to decide this part is at approach or methodogy.
Employee turnover forecasting models is developed using statistical methods. These statistical methods are time series, survival analysis, and data mining methods. Time series methods capture employee turnover seasonal and cyclical pattern and forecast a aggregated turnover number, in term of headcount, by using historical turnover number. Survival analysis identifies the significant internal and external turnover factors and build a Cox PH model to forecast turnover on the individual, departmental, and entity-wide levels. Logistic regression and decision tree methods are tested whether significant factors identified by the literature review are also significant in the studied organization. A set of decision rules are created based on employees' the number of working service years. 
   
The implementation of employee turnover forecast model is also a key part of workforce planning in lean management system. The employee turnover forecasting model is inserted a software program using friendly user interface. The human resource department installs this program, imports the employees information into the program periodically, computes the foretasting information and exports the results. HR uses the production planning and budgeting to calculate the total number of employees needed in a future year as the demand, identifies the current number of employees and combines the predicted number of turnover employees. HR finally find out the number of employee required in next year. 
HR either modifies the employee retention and promotion strategies on the target employees with high turnover probabilities to reduced the employee turnover or prepares to hire new employees to avoid productivity reduction and organization malfunction of a lean management system causing by the employee turnover.



%\begin{figure}
%	\centering
%	\includegraphics[scale=0.45]{model.png}
%	\caption{A Conceptual Model for Implementing Employee Turnover Forecast Model to Lean Management System}
%	\label{fig:model}
%\end{figure}
\section{Outline}
This desertification includes five chapters. Chapter 1 is the introduction of employee turnover forecasting; Chapter 2 introduced time series methods and identified an optimal time series model to forecast the monthly employee turnover number. Chapter 3 emphases how to build a Cox proportional model to identify the significant demographic, organizational factors, and financial indexes, and to forecast employee retirement and voluntary quitting on individual level. Chapter 4 tested six hypothesis regarding to employee turnover significant factors for a research and development organization and generated splitting rules based on those factors to forecast the average number of service years of a certain employee group. Chapter 5 summarizes the findings, provides the implementation of forecasting model and future work.

