\chapter{Conclusions} \label{ch:conclusion}
\section{Summary of findings}
\section{Implementation of Employee Turnover forecasting Models}

 employee turnover model every 5 years and repeat the above steps. Figure \ref{fig:model} shows the conceptual model implementing employee turnover forecast model to  The conceptual models includes three modules: model installation and update, forecast result identification, and strategy generation. These three modules are also discussed in each step. 
 
First, install the employee forecasting model to human resource management system. This step is a practical step and need specify HR staff to involve. For example, after the model installing in the system. HR administrators extract the quarterly or yearly employee turnover records from HR management system which includes employees' individual information rather than a number, such as, gender, age, year of service, department, job title and other kinds of information. Then, input these information to the model and compute the turnover probabilities and aggregated these probabilities as predicted total turnover number in the future. The high predicted probability employees are the target group for HR departement to determine retention and promotion strategies. The aggregated predicted turnover number can be reported to HR manager and certain department directors to prepare for organization budgeting and planning. 

Second, modify the employee retention strategies and build a talent inventory control system. \citet{moncarz2009} found that the effective retention strategies like promotions and training can reduce the employee turnover in long term and positive influence employees retention and tenure. The turnover attributes provide manager and strategy makers an objective, effective, and clear direction for the retention strategies making. The strategies can be programs, polices, and practices in such areas: corporate culture and communication, work environment and job design, promotions, customer contentedness and employee recognition, rewards and compensation \citep{moncarz2009}. In some organizations with good retention plan, the workforce has long tenure in general, but they may still experience high employee turnover, causing by high proportion of aging workforce or voluntary quitting of newly hired employee. Generating the optimal employee inventory management strategy is the solution for reducing the cost and maintain the normal operation of lean management system. Many inventory models can be used to employee inventory management, such as Economic Order Quantity model (EOQ). EOQ model assumes the demand for the employee inventory occurs at a constant rate; an ordering and setup cost $k$ is incurred, when hiring new employees; the cost per-year of holding employee inventory is $h$; and no shortage of employees are allowed. In realistic, the employee demand number can be achieved by the difference between the employee demand and the employee turnover. The company production plan and strategy determines the total number employee required and the employee turnover forecasting model can provide the turnover number in the future one year. Based on EOQ model, the economic order quantity can be determined from mathematical model, which is the economic hiring number in a period. As a result, the optimal hiring strategy can be identified from the inventory model to keep the employee inventory in certain levels. 

The final step is updating the employee turnover forecasting model in at least five years. Because the model is built based on historical data, the internal and external factors influenced the employee turnover can be changed after 5 years. The internal factors can be retention strategy, organization structure, culture and other factor related employee turnover. The external factors might be economic situations like job market and stock market. For example, when there are a lot of opening in the job market, that will cause the increase number of voluntary quit. Also,  if the stock index goes up, that will be an incentive for employee retirement. Therefore, the forecasting result of employee turnover and significant attributes would not be correct. The procedure to update the model can be input the data with all the available variables into the forecasting model, and rerun it to identify the significant attributes. Then apply the significant attributes to the model again to generate parameters for forecasting employee turnover number in the future. If there are no significant attributes or the forecasting number is quite different from actual turnover number. The model has to be re-identified by the professional staff.  

In conclusion, the employee turnover model is a necessary part or function in the lean management system, which provides support for the normal operation of lean management system and reduce the cost/waster due to employee turnover. Applying this model to lean management system needs the cooperation from HR staff, HR managers, department managers, and also statistical staff. The managers should consider this as a part of lean system and provide enough supports, which are the same important as Kanban, 5S, and other lean tools. The employee turnover forecasting model are not limited in manufacturing organization, it could apply to service and government organization as well. Actually, there are many employee turnover studies from health care area in the literature. The future work can be related to analyze which kind of retention strategy is more effective with optimal cost, or which is the suitable model for achieving EOQ. 
% Turnover forecasting two dementions:1, performance of the organzation 
\section{Future Work}
