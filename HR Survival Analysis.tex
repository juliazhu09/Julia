\documentclass[12pt,letterpaper]{article}
\usepackage{amsmath}
\usepackage{amssymb}
\usepackage[round]{natbib}
\usepackage[left =1.0in,right=1.0in,top=1.0in,bottom=1.0in]{geometry}
\author{Xiaojuan Zhu} 
\title{Proposal for Employee Turnover Analysis}
\begin{document}
	\maketitle
	
	%% referece web page: http://merkel.zoneo.net/Latex/natbib.php
	%% \citet{jon90} 	    -->    	Jones et al. (1990)
	%% \citet[chap. 2]{jon90} 	    -->    	Jones et al. (1990, chap. 2)
	%% \citep{jon90} 	    -->    	(Jones et al., 1990)
	%% \citep[chap. 2]{jon90} 	    -->    	(Jones et al., 1990, chap. 2)
	%% \citep[see][]{jon90} 	    -->    	(see Jones et al., 1990)
	%% \citep[see][chap. 2]{jon90} 	    -->    	(see Jones et al., 1990, chap. 2)
	%% \citet*{jon90} 	    -->    	Jones, Baker, and Williams (1990)
	%% \citep*{jon90} 	    -->    	(Jones, Baker, and Williams, 1990)
	
	\section{Introduction }
	Employee turnover is a topic that has drawn the attention of management researchers and practitioners for decades, because employee turnover cost impacts both the operational capabilities and the budget of an organization. Employee turnover is both costly and disruptive to the functioning of most organizations \citep{kacmar2006, mueller1989,staw1980}, and both private firms and governments spend billions of dollars every year managing the issue according to \citet{leonard2001}. %check reference
	The cost for turnover involves recruiting, selecting, training and developing \citep{mobley1982, staw1980}. According to the estimation from U.S. Department of Labor, turnover costs a company one third of a new hire's annual salary to % check this sentence
	replace an employee, which is about \$500 to \$1500 per person for fast-food industry and \$3000 to \$5000 per person for trucking industry to replace a new employee \citep{white1995}. Furthermore, turnover also disrupts the social and communication structures, and causes the productivity loss due to the replacement \citep{mobley1982}. Beyond the replacement cost and operational disruption, turnover demoralizes the attitudes of remaining employees and leads to additional turnover \citep{staw1980}. Therefore, understanding and forecasting turnover at firm and departmental levels is essential for reducing it \citep{kacmar2006} and for effective planning, budgeting, and recruiting in the human resource field. 
	
	Many studies have proved that employee turnover has negative effects with organizational performance. 
	
	
	
	
	
	
	
	\bibliographystyle{abbrvnat}%Choose a bibliograhpic style%
	\bibliography{Bib}
	
\end{document}

