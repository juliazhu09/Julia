\documentclass[12pt,letterpaper]{article}
\usepackage{amsmath}
\usepackage{amssymb}
\usepackage[round]{natbib}
\usepackage[left =1.0in,right=1.0in,top=1.0in,bottom=1.0in]{geometry}
%opening
\title{}
\author{Xiaojuan Zhu}

\begin{document}

\maketitle

\begin{abstract}


\end{abstract}
%3) Outline for the paper.
%1. Introduction - What are the problems and key questions that you are trying to solve? How are you approaching it? What methods?  
%2. Literature review - What has been done before and published in academic or trade literature?  How have people discussed motivations for retirement and quitting in different industries.  What factors are in play?  How does solving this problem help companies?
%3. Data Description and Preparation
%4. Model Development and description
%5. Analysis of Modeling results
%6. Conclusions and managerial implications 
%If we can do 1,3,4,and 5 we may get Tim to help us with 2.

\section{Introduction}
Employee turnover is a topic that has drawn the attention of management researchers and practitioners for decades, because  employee turnover is both costly and disruptive to the functioning of most organizations \citep{staw1980, mueller1989,kacmar2006}, and both private firms and governments spend billions of dollars every year managing the issue according to \citet{leonard2001}. Therefore, understanding the causes of turnover: retirement and voluntary quit, examining the internal and external impacts, effectively forecasting the turnover by these two causes, and measuring the effectiveness and to what extent of the HR policy at firm and departmental levels are the key questions in this study for reducing it and for effective planning, budgeting, and recruiting in the human resource filed. As a funded research project, a large organizational secondary dataset including 12-year employees demographic information and records is transformed, analyzed and modeled by Cox proportional hazard regression models using competing risks analysis to examine the statistically significant factors and to predict employees' conditional retiring and voluntary quitting probabilities. The dataset are also employed to logistic regression and time series models for compare the performance of cox proportional hazard model.This study also examines the forecasting capability of Cox proportional hazard model on the data with two kinds of bias (left truncation and right censor) by simulation.      


%Employee turnover cost impacts both the operational capabilities and the budget of an organization. The cost for turnover involves recruiting, selecting, training and developing \citep{mobley1982, staw1980}. According to the estimation from U.S. Department of Labor, turnover costs a company one third of a new hire's annual salary to 
%replace an employee, which is about \$500 to \$1500 per person for fast-food industry and \$3000 to \$5000 per person for trucking industry \citep{white1995}. Furthermore, turnover also disrupts the social and communication structures, and causes the productivity loss due to the replacement \citep{mobley1982}. Beyond the these cost and operational disruption, turnover demoralizes the attitudes of remaining employees and leads to additional turnover \citep{staw1980}. Therefore, understanding and forecasting turnover at firm and departmental levels is essential for reducing it \citep{kacmar2006} and for effective planning, budgeting, and recruiting in the human resource field. 

%this study is to forecast employee turnover in organizational level using time series and individual level using survival analysis, to examine the internal and external factors contributing on employee turnover, to identify why employee turnover, and to measure the effect of human resource policy on employee turnover based on employee demographic dataset. 

\section{Literature Review}
\section{Data Preparation}
1. data desicribtion, 
2. two data bias, left trucation and right censor
3. policy : counting process.
 
\section{Model Development and Description}
\section{Conclusions and Managerial Implications} 


	\bibliographystyle{abbrvnat}%Choose a bibliograhpic style%
	\bibliography{Bib}
\end{document}
